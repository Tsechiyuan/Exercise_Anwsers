\documentclass[UTF8]{ctexart}[a4paper,12pt]
\usepackage[thmmarks]{ntheorem}
\usepackage{amsmath}

\usepackage{amsfonts,amssymb} 
\usepackage{thmtools}
\usepackage[hmargin=2.5cm,vmargin=2.5cm]{geometry}
\usepackage{tikz-cd,tikz}
\usepackage{graphicx,float}
\usepackage{fancyhdr}
\usepackage{fourier-orns}
\usepackage{quiver}
\usepackage{mathrsfs}

%声明环境
\theorembodyfont{\rmfamily}
\newtheorem{example}{例}[section]              
\newtheorem{algorithm}{算法}[subsection]
\newtheorem{theorem}{定理}[section]            
\newtheorem{definition}{定义}[section]
\newtheorem{axiom}{公理}[section]
\newtheorem{property}{性质}[section]
\newtheorem{proposition}{命题}[section]
\newtheorem{lemma}[theorem]{引理}
\newtheorem{corollary}[theorem]{推论}
{
    \theoremheaderfont{\sffamily}
    \newtheorem*{remark}{注解} 
}
\newtheorem{condition}{条件}
\newtheorem{conclusion}{结论}[section]
\newtheorem{assumption}{假设}
{
\theoremstyle{nonumberplain}
\theoremheaderfont{\bfseries}
\theorembodyfont{\normalfont}
\theoremsymbol{\mbox{$\Box$}}
\newtheorem{proof}{证明}
}
%定义命令
\def\N{\mathbb{N}}
\def\Z{\mathbb{Z}}
\def\Q{\mathbb{Q}}
\def\R{\mathbb{R}}
\def\C{\mathbb{C}}
\def\S{\mathbb{S}}
\def\D{\mathbb{D}}
\def\cC{\mathcal{C}}
\newcommand{\pa}[1]{\dfrac{\partial}{\partial #1}}

%页眉设计
\renewcommand 
\headrule{
\hrulefill
\raisebox{-2.1pt}
{\quad{\FourierOrns M T S N}\quad}
\hrulefill}
\pagestyle{fancy}
%极限余极限
\makeatletter
\newcommand{\Colim@}[2]{
  \vtop{\m@th\ialign{##\cr
    \hfil$#1\operator@font lim$\hfil\cr
    \noalign{\nointerlineskip\kern1.5\ex@}#2\cr
    \noalign{\nointerlineskip\kern-\ex@}\cr}}%
}
\newcommand{\Colim}{%
  \mathop{\mathpalette\Colim@{\rightarrowfill@\scriptscriptstyle}}\nmlimits@
}
\makeatother

\makeatletter
\newcommand{\Lim@}[2]{%
  \vtop{\m@th\ialign{##\cr
    \hfil$#1\operator@font lim$\hfil\cr
    \noalign{\nointerlineskip\kern1.5\ex@}#2\cr
    \noalign{\nointerlineskip\kern-\ex@}\cr}}%
}
\newcommand{\Lim}{%
  \mathop{\mathpalette\Lim@{\leftarrowfill@\scriptscriptstyle}}\nmlimits@
}
\makeatother


\makeatletter
\newcommand{\colim@}[2]{%
  \vtop{\m@th\ialign{##\cr
    \hfil$#1\operator@font oli~$\hfil \cr
    \noalign{\nointerlineskip\kern1.5\ex@}#2\cr
    \noalign{\nointerlineskip\kern-\ex@}\cr}}%
}
\newcommand{\colim}{%
  \mathop{\mathrm{c}\mathpalette\colim@{\rightarrowfill@\scriptscriptstyle}\mathrm{\!\!m}}\nmlimits@
}
\makeatother

\makeatletter
\newcommand{\cone@}[1]{%
  \vtop{\m@th\ialign{##\cr
    \hfil$#1\operator@font cone$\hfil\cr
    \noalign{\nointerlineskip\kern1.5\ex@}\cr
    \noalign{\nointerlineskip\kern-\ex@}\cr}}%
}
\newcommand{\cone}{%
  \mathop{\mathpalette\cone@{\scriptscriptstyle}}\nmlimits@
}
\makeatother

%超链接红色
\usepackage[colorlinks,linkcolor=red]{hyperref}

\usepackage{enumerate}
\numberwithin{equation}{section}


\title{Do Carmo黎曼几何习题}
\author{颜成子游}
\begin{document}
\maketitle
\tableofcontents
\section{第一章}
\section{第二章}
1.设$c(t)$的切向量为$\dot{c}(t)$。根据平行移动可知:
\begin{align}
  \nabla_{\dot{c}(t)} P_{c,t_0,t}=0
\end{align}
等距性:
\begin{align}
  s(t)=\|P_{c,t_0,t}\|^2, \dot{s}(t)=0 \Rightarrow s(t)\equiv s(t_0)
\end{align}
保定向:

设$e_i$是$t_0$处的一组单位正交定向基,则$P_{c,t_0,t}(e_i)$是一组$t$处的单位正交基。这两组基诱导的定向必须连续变化,于是$P$是保定向的映射。

2.
\begin{align}
  \frac{d}{dt}P^{-1}_{c,t_0,t}(Y(c(t)))|_{t=t_0}=\lim_{t \to t_0}\frac{Y(c(t))-P_{c,t_0,t}(Y(p))}{t-t_0}=\nabla_X (Y-P_{c,t_0,t}Y)(p)=\nabla_X Y(p)
\end{align}

3.$\nabla$是联络是平凡的。我们需要说明这是黎曼联络。

根据拉回,$M$的度量是$f^*g$.设$f_*X=\bar{X}$。(其他向量场也同理)。从而:
\begin{align}
  X\langle Y,Z\rangle=X\langle f_* Y,f_*Z\rangle=f_*(X) \langle f_* Y,f_*Z\rangle=\langle \bar{\nabla}_{\bar{X}}\bar{Y},\bar{Z}\rangle+\langle \bar{Y},\bar{\nabla}_{\bar{X}}\bar{Z}\rangle=\langle \nabla_X Y,Z\rangle+\langle Y,\nabla_X Z\rangle
\end{align}

4.(a)借鉴上题的思路。设$\nabla$是$M$上的联络,则$\nabla_V V=0$.于是$\R^3$上的平凡联络$D$满足:
\begin{align}
  D_V V \perp V
\end{align}
若上述公式成立,则$\nabla_V V=0$。所以$V$沿着曲线平行移动。

(b)

5.欧氏空间上平行移动与点无关。因为欧氏空间的联络是平凡的,从而对单位向量$e_i$求联络总是$0$.

若不是欧氏空间,则可以举球面$S^2$的例子。从北极点平行移动到南极点,走不同经线得到的结果不同。

6.

7.

8.(a)略。带入公式:
\begin{align}
  \Gamma_{ij}^k=\frac{1}{2}\sum_m \{\frac{\partial}{\partial x_i}g_{jm}+\frac{\partial}{\partial x_j}g_{im}-\frac{\partial}{\partial x_m}g_{ij}\}g^{km}
\end{align}即可。

(b)使用Christoffel记号计算平行移动的方程即可。

9.(a)仿照Levi-Civita联络的证明即可。

因为联络和伪黎曼度量是相容的,且联络是无挠的,我们仍然有:
\begin{align}
  X\langle Y,Z\rangle+Y\langle Z,X\rangle-Z\langle X,Y\rangle=\langle [X,Z],Y \rangle+\langle [Y,Z],X \rangle+\langle [X,Y],Z \rangle+2\langle Z,\nabla_Y X \rangle
\end{align}
因为$\langle\cdot,\cdot \rangle$仍然是非退化的,所以通过指定$\langle Z,\nabla_Y X\rangle$的值,我们仍然可以给出$\nabla$的唯一性和存在性。

(b)设洛伦兹度量下的Levi-Civita联络是$\nabla$.平凡度量下的Levi-Civita联络是$D$.我们说明若$\nabla_X Y=0$与$D_X Y=0$等价。

对于$\nabla$而言,带入式(7),不难发现$\nabla_{\frac{\partial}{\partial x_j}}\frac{\partial}{\partial x_i}=0,0\leq i,j \leq n$.所以$\nabla_{X^i \frac{\partial}{\partial x_i}}(Y^j \frac{\partial}{\partial x_j})=0$等价于:
\begin{align}
  X^i (\nabla_{\frac{\partial}{\partial x_i}}Y^j)\frac{\partial}{\partial x_j}=0
\end{align}
由于任何联络作用在函数上总是求李导数,所以上述方程可以直接替换为:
\begin{align}
  X^i (D_{\frac{\partial}{\partial x_i}}Y^j)\frac{\partial}{\partial x_j}=0
\end{align}
于是$D_X Y=0$与$\nabla_X Y=0$等价。
\newpage
\section{第三章}
1.(Geodesic of a surface of revolution)

(a)计算:
\begin{align}
  \langle \pa{v},\pa{u}\rangle=0, \langle \pa{v},\pa{v}\rangle=f'^2+g'^2,\langle \pa{u},\pa{u}\rangle=f^2
\end{align}

(b)设测地线$\gamma(t)$为$(u(t),v(t))$.则切向量为:
\begin{align}
  \dot{\gamma}(t)=u'(t)\pa{u}+v'(t)\pa{v}
\end{align}
显然需要先计算联络系数。我们有:
\begin{align}
  &\Gamma_{11}^1=0,\Gamma_{12}^1=\Gamma_{21}^1=\frac{f'}{f},\Gamma_{22}^1=0\\
  &\Gamma_{11}^2=-\frac{ff'}{f'^2+g'^2},\Gamma_{12}^2=\Gamma_{21}^2=0,\Gamma_{22}^2=\frac{f'f''+g'g''}{f'^2+g'^2}
\end{align}
带入测地线方程:
\begin{align}
  &\frac{d^2u}{dt^2}+2\frac{f'}{f}\frac{du}{dt}\frac{dv}{dt}=0\\
  &\frac{d^2v}{dt^2}-\frac{ff'}{(f')^2+(g')^2}(\frac{du}{dt})^2+\frac{f'f''+g'g''}{f'^2+g'^2}(\frac{dv}{dt})^2=0
\end{align}

(c)$|\gamma'(t)|^2=u'^2f^2+v'^2(f'^2+g'^2)$.对其求导:
\begin{align}
  \frac{d}{dt}|\gamma'(t)|^2&=2u'u''f^2+2u'^2ff'v'+2v''v'(f'^2+g'^2)+2v'^3(f''f'+g''g')\\&=2u'u''f^2+2u'^2ff'v'+2v'[ff'u'^2-(f'f''+g'g'')v'^2]+2v'^3(f''f'+g''g')\\&=2fu'(u''f+2u'v'f')=0
\end{align}
略去第二个有向角的计算。记录为:
\begin{align}
  r\cos \beta=\rm{const}
\end{align}

(d)我认为是一个错题。

2.(定义切丛上的Riemann度量)

(a)对于给出的度量表达式,良定义意为选择的曲线$p,v,q,w$并不影响计算的结果.其次,这是一个对称的正定二次型。

对称的正定二次型是平凡的。因而我们只需要考虑良定义问题。观察表达式:
\begin{align}
  \langle V,W\rangle_{p,v}=\langle d\pi(V),d\pi(W)\rangle_p+\langle \frac{Dv}{dt}(0),\frac{Dw}{ds}(0)\rangle_p
\end{align}

显然第一项是自然良定的。对于第二项,注意到:
\begin{align}
  \frac{Dv}{dt}(0)=\nabla_{d\pi(V)} v(t),\frac{Dw}{ds}(0)=\nabla_{d\pi(W)} w(s)
\end{align}
我们需要说明$\nabla_{d\pi(V)} v(t)$是不依赖$v(t)$选取的向量。不妨根据联络的定义展开:
\begin{align}
  \nabla_{d\pi(V)} v(t)=((d\pi(V))^j\frac{\partial v^i}{\partial x^j}+\Gamma_{kj}^i(d\pi(V))^kY^j)\pa{x^i},\frac{\partial v^i}{\partial x^j}=V^{n+i}
\end{align}
于是该向量被$v$,$V$所表达,因而是良定义的。

(b)纤维上$d\pi(W)=0$。因此$V$是水平向量等价于:
\begin{align}
  \langle \frac{Dv}{dt}(0),\frac{Dw}{ds}(0)\rangle\equiv0,\forall w(t)
\end{align}
因此$\frac{Dv}{dt}=\nabla_{\dot{p}(t)}v(t)=0$恒成立,即$v(t)$沿着$p(t)$平行移动。

(c)设$v(t)$是测地向量场($M$上的)。这也可以写作$v:M \to TM$表。我们断言$v_*v(t)$是水平向量场。不妨设$v(t)$对应的测地线是$p(t)$。

于是在$(p(t),v(t))$处,$\nabla_{\dot{p(t)}}v(t)=0$恒成立,从而$v_*v(t)$是水平向量场。

(d)设$\bar{\alpha}(t)=(\alpha(t),v(t))$,其中$v(t)$是沿着$\alpha(t)$的向量场。则:
\begin{align}
  \|\dot{\bar{\alpha}}(t)\|^2=\|\dot{\alpha}(t)\|^2+\|\frac{Dv}{dt}\|^2 \geq \|\dot{\alpha}(t)\|^2
\end{align}
于是我们有:
\begin{align}
  l(\bar{\alpha})\geq l(\alpha)
\end{align}
若$\alpha(t)$是测地线且$v(t)$是$\alpha(t)$的切向量,我们有:
\begin{align}
  \|\dot{\bar{\alpha}}(t)\|^2=\|\dot{\alpha}(t)\|^2+\|\frac{Dv}{dt}\|^2 = \|\dot{\alpha}(t)\|^2+0= \|\dot{\alpha}(t)\|^2
\end{align}
于是$l(\alpha)=l(\bar{\alpha})$。

现在考虑一个能使得测地线是最短线的凸邻域。于是$\bar{\alpha}$成为了所有$\bar{\gamma}(t)$中的最短线,因而是测地线。

(e)因为$\dfrac{Dw}{dt}=0$($W$水平),所以第一个等式成立。

如果$W$垂直,则$d\pi(W)=0$且$\dfrac{Dw}{dt}$退化为$W$.所以第二个等式成立。
\end{document}



