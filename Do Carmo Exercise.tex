\documentclass[UTF8]{ctexart}[a4paper,12pt]
\usepackage[thmmarks]{ntheorem}
\usepackage{amsmath}
\usepackage{amsfonts,amssymb} 
\usepackage{thmtools}
\usepackage[hmargin=2.5cm,vmargin=2.5cm]{geometry}
\usepackage{tikz-cd,tikz}
\usepackage{graphicx,float}
\usepackage{fancyhdr}
\usepackage{fourier-orns}
\usepackage{quiver}
\usepackage{mathrsfs}

%声明环境
\theorembodyfont{\rmfamily}
\newtheorem{example}{例}[section]              
\newtheorem{algorithm}{算法}[subsection]
\newtheorem{theorem}{定理}[section]            
\newtheorem{definition}{定义}[section]
\newtheorem{axiom}{公理}[section]
\newtheorem{property}{性质}[section]
\newtheorem{proposition}{命题}[section]
\newtheorem{lemma}[theorem]{引理}
\newtheorem{corollary}[theorem]{推论}
{
    \theoremheaderfont{\sffamily}
    \newtheorem*{remark}{注解} 
}
\newtheorem{condition}{条件}
\newtheorem{conclusion}{结论}[section]
\newtheorem{assumption}{假设}
{
\theoremstyle{nonumberplain}
\theoremheaderfont{\bfseries}
\theorembodyfont{\normalfont}
\theoremsymbol{\mbox{$\Box$}}
\newtheorem{proof}{证明}
}
%定义命令
\def\N{\mathbb{N}}
\def\Z{\mathbb{Z}}
\def\Q{\mathbb{Q}}
\def\R{\mathbb{R}}
\def\C{\mathbb{C}}
\def\S{\mathbb{S}}
\def\D{\mathbb{D}}
\def\H{\mathbb{H}}
\def\cC{\mathcal{C}}

%页眉设计
\renewcommand 
\headrule{
\hrulefill
\raisebox{-2.1pt}
{\quad{\FourierOrns M T S N}\quad}
\hrulefill}
\pagestyle{fancy}
%极限余极限
\makeatletter
\newcommand{\Colim@}[2]{
  \vtop{\m@th\ialign{##\cr
    \hfil$#1\operator@font lim$\hfil\cr
    \noalign{\nointerlineskip\kern1.5\ex@}#2\cr
    \noalign{\nointerlineskip\kern-\ex@}\cr}}%
}
\newcommand{\Colim}{%
  \mathop{\mathpalette\Colim@{\rightarrowfill@\scriptscriptstyle}}\nmlimits@
}
\makeatother

\makeatletter
\newcommand{\Lim@}[2]{%
  \vtop{\m@th\ialign{##\cr
    \hfil$#1\operator@font lim$\hfil\cr
    \noalign{\nointerlineskip\kern1.5\ex@}#2\cr
    \noalign{\nointerlineskip\kern-\ex@}\cr}}%
}
\newcommand{\Lim}{%
  \mathop{\mathpalette\Lim@{\leftarrowfill@\scriptscriptstyle}}\nmlimits@
}
\makeatother


\makeatletter
\newcommand{\colim@}[2]{%
  \vtop{\m@th\ialign{##\cr
    \hfil$#1\operator@font oli~$\hfil \cr
    \noalign{\nointerlineskip\kern1.5\ex@}#2\cr
    \noalign{\nointerlineskip\kern-\ex@}\cr}}%
}
\newcommand{\colim}{%
  \mathop{\mathrm{c}\mathpalette\colim@{\rightarrowfill@\scriptscriptstyle}\mathrm{\!\!m}}\nmlimits@
}
\makeatother

\makeatletter
\newcommand{\cone@}[1]{%
  \vtop{\m@th\ialign{##\cr
    \hfil$#1\operator@font cone$\hfil\cr
    \noalign{\nointerlineskip\kern1.5\ex@}\cr
    \noalign{\nointerlineskip\kern-\ex@}\cr}}%
}
\newcommand{\cone}{%
  \mathop{\mathpalette\cone@{\scriptscriptstyle}}\nmlimits@
}
\makeatother

%超链接红色
\usepackage[colorlinks,linkcolor=red]{hyperref}

\usepackage{enumerate}


\title{Do Carmo黎曼几何习题}
\author{颜成子游}
\begin{document}
\maketitle
\tableofcontents
\section{第一章}
\section{第二章}
1.设$c(t)$的切向量为$\dot{c}(t)$。根据平行移动可知:
\begin{align}
  \nabla_{\dot{c}(t)} P_{c,t_0,t}=0
\end{align}
等距性:
\begin{align}
  s(t)=\|P_{c,t_0,t}\|^2, \dot{s}(t)=0 \Rightarrow s(t)\equiv s(t_0)
\end{align}
保定向:

设$e_i$是$t_0$处的一组单位正交定向基,则$P_{c,t_0,t}(e_i)$是一组$t$处的单位正交基。这两组基诱导的定向必须连续变化,于是$P$是保定向的映射。

2.
\begin{align}
  \frac{d}{dt}P^{-1}_{c,t_0,t}(Y(c(t)))|_{t=t_0}=\lim_{t \to t_0}\frac{Y(c(t))-P_{c,t_0,t}(Y(p))}{t-t_0}=\nabla_X (Y-P_{c,t_0,t}Y)(p)=\nabla_X Y(p)
\end{align}

3.$\nabla$是联络是平凡的。我们需要说明这是黎曼联络。

根据拉回,$M$的度量是$f^*g$.设$f_*X=\bar{X}$。(其他向量场也同理)。从而:
\begin{align}
  X\langle Y,Z\rangle=X\langle f_* Y,f_*Z\rangle=f_*(X) \langle f_* Y,f_*Z\rangle=\langle \bar{\nabla}_{\bar{X}}\bar{Y},\bar{Z}\rangle+\langle \bar{Y},\bar{\nabla}_{\bar{X}}\bar{Z}\rangle=\langle \nabla_X Y,Z\rangle+\langle Y,\nabla_X Z\rangle
\end{align}

4.(a)借鉴上题的思路。设$\nabla$是$M$上的联络,则$\nabla_V V=0$.于是$\R^3$上的平凡联络$D$满足:
\begin{align}
  D_V V \perp V
\end{align}
若上述公式成立,则$\nabla_V V=0$。所以$V$沿着曲线平行移动。

(b)

5.欧氏空间上平行移动与点无关。因为欧氏空间的联络是平凡的,从而对单位向量$e_i$求联络总是$0$.

若不是欧氏空间,则可以举球面$S^2$的例子。从北极点平行移动到南极点,走不同经线得到的结果不同。

6.

7.

8.(a)略。带入公式:
\begin{align}
  \Gamma_{ij}^k=\frac{1}{2}\sum_m \{\frac{\partial}{\partial x_i}g_{jm}+\frac{\partial}{\partial x_j}g_{im}-\frac{\partial}{\partial x_m}g_{ij}\}g^{km}
\end{align}即可。

(b)使用Christoffel记号计算平行移动的方程即可。

9.(a)仿照Levi-Civita联络的证明即可。

因为联络和伪黎曼度量是相容的,且联络是无挠的,我们仍然有:
\begin{align}
  X\langle Y,Z\rangle+Y\langle Z,X\rangle-Z\langle X,Y\rangle=\langle [X,Z],Y \rangle+\langle [Y,Z],X \rangle+\langle [X,Y],Z \rangle+2\langle Z,\nabla_Y X \rangle
\end{align}
因为$\langle\cdot,\cdot \rangle$仍然是非退化的,所以通过指定$\langle Z,\nabla_Y X\rangle$的值,我们仍然可以给出$\nabla$的唯一性和存在性。

(b)设洛伦兹度量下的Levi-Civita联络是$\nabla$.平凡度量下的Levi-Civita联络是$D$.我们说明若$\nabla_X Y=0$与$D_X Y=0$等价。

对于$\nabla$而言,带入式(7),不难发现$\nabla_{\frac{\partial}{\partial x_j}}\frac{\partial}{\partial x_i}=0,0\leq i,j \leq n$.所以$\nabla_{X^i \frac{\partial}{\partial x_i}}(Y^j \frac{\partial}{\partial x_j})=0$等价于:
\begin{align}
  X^i (\nabla_{\frac{\partial}{\partial x_i}}Y^j)\frac{\partial}{\partial x_j}=0
\end{align}
由于任何联络作用在函数上总是求李导数,所以上述方程可以直接替换为:
\begin{align}
  X^i (D_{\frac{\partial}{\partial x_i}}Y^j)\frac{\partial}{\partial x_j}=0
\end{align}
于是$D_X Y=0$与$\nabla_X Y=0$等价。
\end{document}



