\documentclass[UTF8]{ctexart}[a4paper,12pt]
\usepackage[thmmarks]{ntheorem}
\usepackage{amsmath}

\usepackage{amsfonts,amssymb} 
\usepackage{thmtools}
\usepackage[hmargin=2.5cm,vmargin=2.5cm]{geometry}
\usepackage{tikz-cd,tikz}
\usepackage{graphicx,float}
\usepackage{fancyhdr}
\usepackage{fourier-orns}
\usepackage{quiver}
\usepackage{mathrsfs}

%声明环境
\theorembodyfont{\rmfamily}
\newtheorem{example}{例}[section]              
\newtheorem{question}{题}[section]
\newtheorem{theorem}{定理}[section]            
\newtheorem{lemma}[theorem]{引理}

\theoremstyle{nonumberplain}
\theoremheaderfont{\bfseries}
\theorembodyfont{\normalfont}
\theoremsymbol{\mbox{$\Box$}}
\newtheorem{proof}{证明}
\newtheorem{solution}{解}


%定义命令
\def\N{\mathbb{N}}
\def\Z{\mathbb{Z}}
\def\Q{\mathbb{Q}}
\def\R{\mathbb{R}}
\def\C{\mathbb{C}}
\def\S{\mathbb{S}}
\def\D{\mathbb{D}}
\def\cC{\mathcal{C}}
\newcommand{\pa}[1]{\dfrac{\partial}{\partial #1}}

%页眉设计
\renewcommand 
\headrule{
\hrulefill
\raisebox{-2.1pt}
{\quad{\FourierOrns M T S N}\quad}
\hrulefill}
\pagestyle{fancy}
%极限余极限
\makeatletter
\newcommand{\Colim@}[2]{
  \vtop{\m@th\ialign{##\cr
    \hfil$#1\operator@font lim$\hfil\cr
    \noalign{\nointerlineskip\kern1.5\ex@}#2\cr
    \noalign{\nointerlineskip\kern-\ex@}\cr}}%
}
\newcommand{\Colim}{%
  \mathop{\mathpalette\Colim@{\rightarrowfill@\scriptscriptstyle}}\nmlimits@
}
\makeatother

\makeatletter
\newcommand{\Lim@}[2]{%
  \vtop{\m@th\ialign{##\cr
    \hfil$#1\operator@font lim$\hfil\cr
    \noalign{\nointerlineskip\kern1.5\ex@}#2\cr
    \noalign{\nointerlineskip\kern-\ex@}\cr}}%
}
\newcommand{\Lim}{%
  \mathop{\mathpalette\Lim@{\leftarrowfill@\scriptscriptstyle}}\nmlimits@
}
\makeatother


\makeatletter
\newcommand{\colim@}[2]{%
  \vtop{\m@th\ialign{##\cr
    \hfil$#1\operator@font oli~$\hfil \cr
    \noalign{\nointerlineskip\kern1.5\ex@}#2\cr
    \noalign{\nointerlineskip\kern-\ex@}\cr}}%
}
\newcommand{\colim}{%
  \mathop{\mathrm{c}\mathpalette\colim@{\rightarrowfill@\scriptscriptstyle}\mathrm{\!\!m}}\nmlimits@
}
\makeatother

\makeatletter
\newcommand{\cone@}[1]{%
  \vtop{\m@th\ialign{##\cr
    \hfil$#1\operator@font cone$\hfil\cr
    \noalign{\nointerlineskip\kern1.5\ex@}\cr
    \noalign{\nointerlineskip\kern-\ex@}\cr}}%
}
\newcommand{\cone}{%
  \mathop{\mathpalette\cone@{\scriptscriptstyle}}\nmlimits@
}
\makeatother

%超链接红色
\usepackage[colorlinks,linkcolor=red]{hyperref}

\usepackage{enumerate}
\numberwithin{equation}{section}


\title{Do Carmo黎曼几何习题}
\author{颜成子游}
\begin{document}
\maketitle
\tableofcontents
\section{第一章}
\section{第二章}
1.设$c(t)$的切向量为$\dot{c}(t)$。根据平行移动可知:
\begin{align}
  \nabla_{\dot{c}(t)} P_{c,t_0,t}=0
\end{align}
等距性:
\begin{align}
  s(t)=\|P_{c,t_0,t}\|^2, \dot{s}(t)=0 \Rightarrow s(t)\equiv s(t_0)
\end{align}
保定向:

设$e_i$是$t_0$处的一组单位正交定向基,则$P_{c,t_0,t}(e_i)$是一组$t$处的单位正交基。这两组基诱导的定向必须连续变化,于是$P$是保定向的映射。

2.
\begin{align}
  \frac{d}{dt}P^{-1}_{c,t_0,t}(Y(c(t)))|_{t=t_0}=\lim_{t \to t_0}\frac{Y(c(t))-P_{c,t_0,t}(Y(p))}{t-t_0}=\nabla_X (Y-P_{c,t_0,t}Y)(p)=\nabla_X Y(p)
\end{align}

3.$\nabla$是联络是平凡的。我们需要说明这是黎曼联络。

根据拉回,$M$的度量是$f^*g$.设$f_*X=\bar{X}$。(其他向量场也同理)。从而:
\begin{align}
  X\langle Y,Z\rangle=X\langle f_* Y,f_*Z\rangle=f_*(X) \langle f_* Y,f_*Z\rangle=\langle \bar{\nabla}_{\bar{X}}\bar{Y},\bar{Z}\rangle+\langle \bar{Y},\bar{\nabla}_{\bar{X}}\bar{Z}\rangle=\langle \nabla_X Y,Z\rangle+\langle Y,\nabla_X Z\rangle
\end{align}

4.(a)借鉴上题的思路。设$\nabla$是$M$上的联络,则$\nabla_V V=0$.于是$\R^3$上的平凡联络$D$满足:
\begin{align}
  D_V V \perp V
\end{align}
若上述公式成立,则$\nabla_V V=0$。所以$V$沿着曲线平行移动。

(b)

5.欧氏空间上平行移动与点无关。因为欧氏空间的联络是平凡的,从而对单位向量$e_i$求联络总是$0$.

若不是欧氏空间,则可以举球面$S^2$的例子。从北极点平行移动到南极点,走不同经线得到的结果不同。

6.

7.

8.(a)略。带入公式:
\begin{align}
  \Gamma_{ij}^k=\frac{1}{2}\sum_m \{\frac{\partial}{\partial x_i}g_{jm}+\frac{\partial}{\partial x_j}g_{im}-\frac{\partial}{\partial x_m}g_{ij}\}g^{km}
\end{align}即可。

(b)使用Christoffel记号计算平行移动的方程即可。

9.(a)仿照Levi-Civita联络的证明即可。

因为联络和伪黎曼度量是相容的,且联络是无挠的,我们仍然有:
\begin{align}
  X\langle Y,Z\rangle+Y\langle Z,X\rangle-Z\langle X,Y\rangle=\langle [X,Z],Y \rangle+\langle [Y,Z],X \rangle+\langle [X,Y],Z \rangle+2\langle Z,\nabla_Y X \rangle
\end{align}
因为$\langle\cdot,\cdot \rangle$仍然是非退化的,所以通过指定$\langle Z,\nabla_Y X\rangle$的值,我们仍然可以给出$\nabla$的唯一性和存在性。

(b)设洛伦兹度量下的Levi-Civita联络是$\nabla$.平凡度量下的Levi-Civita联络是$D$.我们说明若$\nabla_X Y=0$与$D_X Y=0$等价。

对于$\nabla$而言,带入式(7),不难发现$\nabla_{\frac{\partial}{\partial x_j}}\frac{\partial}{\partial x_i}=0,0\leq i,j \leq n$.所以$\nabla_{X^i \frac{\partial}{\partial x_i}}(Y^j \frac{\partial}{\partial x_j})=0$等价于:
\begin{align}
  X^i (\nabla_{\frac{\partial}{\partial x_i}}Y^j)\frac{\partial}{\partial x_j}=0
\end{align}
由于任何联络作用在函数上总是求李导数,所以上述方程可以直接替换为:
\begin{align}
  X^i (D_{\frac{\partial}{\partial x_i}}Y^j)\frac{\partial}{\partial x_j}=0
\end{align}
于是$D_X Y=0$与$\nabla_X Y=0$等价。
\newpage
\section{第三章}
1.(Geodesic of a surface of revolution)

(a)计算:
\begin{align}
  \langle \pa{v},\pa{u}\rangle=0, \langle \pa{v},\pa{v}\rangle=f'^2+g'^2,\langle \pa{u},\pa{u}\rangle=f^2
\end{align}

(b)设测地线$\gamma(t)$为$(u(t),v(t))$.则切向量为:
\begin{align}
  \dot{\gamma}(t)=u'(t)\pa{u}+v'(t)\pa{v}
\end{align}
显然需要先计算联络系数。我们有:
\begin{align}
  &\Gamma_{11}^1=0,\Gamma_{12}^1=\Gamma_{21}^1=\frac{f'}{f},\Gamma_{22}^1=0\\
  &\Gamma_{11}^2=-\frac{ff'}{f'^2+g'^2},\Gamma_{12}^2=\Gamma_{21}^2=0,\Gamma_{22}^2=\frac{f'f''+g'g''}{f'^2+g'^2}
\end{align}
带入测地线方程:
\begin{align}
  &\frac{d^2u}{dt^2}+2\frac{f'}{f}\frac{du}{dt}\frac{dv}{dt}=0\\
  &\frac{d^2v}{dt^2}-\frac{ff'}{(f')^2+(g')^2}(\frac{du}{dt})^2+\frac{f'f''+g'g''}{f'^2+g'^2}(\frac{dv}{dt})^2=0
\end{align}

(c)$|\gamma'(t)|^2=u'^2f^2+v'^2(f'^2+g'^2)$.对其求导:
\begin{align}
  \frac{d}{dt}|\gamma'(t)|^2&=2u'u''f^2+2u'^2ff'v'+2v''v'(f'^2+g'^2)+2v'^3(f''f'+g''g')\\&=2u'u''f^2+2u'^2ff'v'+2v'[ff'u'^2-(f'f''+g'g'')v'^2]+2v'^3(f''f'+g''g')\\&=2fu'(u''f+2u'v'f')=0
\end{align}
略去第二个有向角的计算。记录为:
\begin{align}
  r\cos \beta=\rm{const}
\end{align}

(d)我认为是一个错题。

2.(定义切丛上的Riemann度量)

(a)对于给出的度量表达式,良定义意为选择的曲线$p,v,q,w$并不影响计算的结果.其次,这是一个对称的正定二次型。

对称的正定二次型是平凡的。因而我们只需要考虑良定义问题。观察表达式:
\begin{align}
  \langle V,W\rangle_{p,v}=\langle d\pi(V),d\pi(W)\rangle_p+\langle \frac{Dv}{dt}(0),\frac{Dw}{ds}(0)\rangle_p
\end{align}

显然第一项是自然良定的。对于第二项,注意到:
\begin{align}
  \frac{Dv}{dt}(0)=\nabla_{d\pi(V)} v(t),\frac{Dw}{ds}(0)=\nabla_{d\pi(W)} w(s)
\end{align}
我们需要说明$\nabla_{d\pi(V)} v(t)$是不依赖$v(t)$选取的向量。不妨根据联络的定义展开:
\begin{align}
  \nabla_{d\pi(V)} v(t)=((d\pi(V))^j\frac{\partial v^i}{\partial x^j}+\Gamma_{kj}^i(d\pi(V))^kY^j)\pa{x^i},\frac{\partial v^i}{\partial x^j}=V^{n+i}
\end{align}
于是该向量被$v$,$V$所表达,因而是良定义的。

(b)纤维上$d\pi(W)=0$。因此$V$是水平向量等价于:
\begin{align}
  \langle \frac{Dv}{dt}(0),\frac{Dw}{ds}(0)\rangle\equiv0,\forall w(t)
\end{align}
因此$\frac{Dv}{dt}=\nabla_{\dot{p}(t)}v(t)=0$恒成立,即$v(t)$沿着$p(t)$平行移动。

(c)设$v(t)$是测地向量场($M$上的)。这也可以写作$v:M \to TM$表。我们断言$v_*v(t)$是水平向量场。不妨设$v(t)$对应的测地线是$p(t)$。

于是在$(p(t),v(t))$处,$\nabla_{\dot{p(t)}}v(t)=0$恒成立,从而$v_*v(t)$是水平向量场。

(d)设$\bar{\alpha}(t)=(\alpha(t),v(t))$,其中$v(t)$是沿着$\alpha(t)$的向量场。则:
\begin{align}
  \|\dot{\bar{\alpha}}(t)\|^2=\|\dot{\alpha}(t)\|^2+\|\frac{Dv}{dt}\|^2 \geq \|\dot{\alpha}(t)\|^2
\end{align}
于是我们有:
\begin{align}
  l(\bar{\alpha})\geq l(\alpha)
\end{align}
若$\alpha(t)$是测地线且$v(t)$是$\alpha(t)$的切向量,我们有:
\begin{align}
  \|\dot{\bar{\alpha}}(t)\|^2=\|\dot{\alpha}(t)\|^2+\|\frac{Dv}{dt}\|^2 = \|\dot{\alpha}(t)\|^2+0= \|\dot{\alpha}(t)\|^2
\end{align}
于是$l(\alpha)=l(\bar{\alpha})$。

现在考虑一个能使得测地线是最短线的凸邻域。于是$\bar{\alpha}$成为了所有$\bar{\gamma}(t)$中的最短线,因而是测地线。

(e)因为$\dfrac{Dw}{dt}=0$($W$水平),所以第一个等式成立。

如果$W$垂直,则$d\pi(W)=0$且$\dfrac{Dw}{dt}$退化为$W$.所以第二个等式成立。

\quad

3.设$G$是李群,$\mathcal{G}$是李代数.给定$X \in \mathcal{G}$,有积分曲线:
\begin{align}
  \varphi:(-\epsilon,+\epsilon) \to G,\varphi(0)=e,\varphi'(t)=X(\varphi(t))
\end{align}

(a)设$t_0\in (-\epsilon,\epsilon)$且$\varphi(t_0)=y$.根据左不变性,可以推出$t \mapsto y^{-1}\varphi(t)$是在$t_0$处经过$e$的$X$的积分曲线。

事实上,该曲线的切向量场为$L_{*,y^{-1}}X=X$。因此根据积分曲线的唯一性可知$y^{-1}\varphi(t)=\varphi(t-t_0)$.即$\varphi(t_0)^{-1}\varphi(t)=\varphi(t-t_0)$.

由于$t_0$是任意的,所以在$(-\epsilon,\epsilon)$上,我们有$\varphi(t+s)=\varphi(t)\varphi(s)$。用简单的微分方程知识可以推得$t$对于所有$\R$都有定义。

(b)对于$Y \in \mathcal{G}$,需要证明$\nabla_Y Y=0$.

考虑关系:
\begin{align}
  2\langle X,\nabla_Y Y\rangle=2Y\langle X,Y\rangle-X\langle Y,Y\rangle+2\langle Y,[X,Y]\rangle
\end{align}
因为$X,Y$是左不变的向量场,所以$\langle X,Y\rangle$和$\langle Y,Y\rangle$恒定。于是:
\begin{align}
  \langle X,\nabla_Y Y\rangle=\langle Y,[X,Y]\rangle
\end{align}
而度量是双不变的,于是:
\begin{align}
  \langle [U,X],Y \rangle=-\langle U,[V,X]\rangle
\end{align}
上述等式可以参考伴随表示$\rm{Ad}$和$\rm{ad}$之间的关系.

所以$\langle Y,[X,Y]\rangle=0$恒成立。于是$\langle \nabla_Y Y,X\rangle=0$恒成立。于是$\nabla_Y Y=0$.

4.

(a)取满足定理3.7的$W$。对于任意点$p \in W$,有$\exp_p$是$W$上的微分同胚。于是任何点$q$都与$q$有测地线连接,即$W$可缩。

(b)使用$W_p$即可。有限交的情况下用测地线的唯一性可以保证可缩。

5.

(a)设$V$是线性场且$V:\R^n \to \R^n$是矩阵。于是$V$的积分曲线:
\begin{align}
  \dot{x}(t)=Ax(t),x(t)=e^{tA}x(0) 
\end{align}
所以$\varphi(t_0,p)=e^{t_0A}p$.若这个线性变换是等距,则$e^{t_0A} \in O(n)$.即:
\begin{align}
  e^{t_0A}(e^{t_0A})^T=e^{t_0(A+A^T)}=\rm{Id} \Rightarrow A+A^T=0
\end{align}

(b)取$q=\exp_p(v)\neq p,v \in T_pM$.我们说明$\langle X_q,(d\exp_p)_q v\rangle=0$.

取$q$处的$X$的积分曲线$\gamma(t)$,$\gamma(0)=q$.通过选择适当的$U$作为正规坐标系,我们可以假定$\gamma(t) \in U$且$X$在$U$上满足Killing场的定义。

从而存在$v(t)\in T_pM$使得$\exp_p v(t)=\gamma(t)$,$v(0)=v$.

于是$X_q=\dot{\gamma}(0)=\dot{\exp_pv}(0)$.因为$\exp_p$是微分同胚,于是$X_q=d(\exp_p)_q \dot{v}(0)$.

而根据$X$是无穷小等距可知:
\begin{align}
  \|v(t)\| \equiv \mathrm{Const} \Rightarrow \langle v(t),\dot{v}(t)\rangle_p=0
\end{align}

(c)设$X$在$p$处生成的积分曲线是$\varphi_p$.我们断言$Y$在$f(p)$生成的积分曲线是$f \circ \varphi_p$.

事实上:
\begin{align}
  \dot{f\circ\varphi_p}(t)=f_{*,\varphi(t)}(\dot{\varphi_p}(t))=f_{*,\varphi(t)}(X(\varphi(t)))=Y(f(\varphi(t)))
\end{align}

$f$是等距,意味着$\varphi$是等距与$f\circ \varphi$是等距是等价的。

(d)(killing equation)
$X$是killing field当且仅当对于任何向量场$Y,Z$,有:
\begin{align}
  \langle \nabla_Y X,Z\rangle+\langle \nabla_Z X,Y\rangle=0
\end{align}

我们说明上述结果。只需要对$X(p)\neq 0$的地方说明。设$U$是$p$处的一个正规坐标系,$S$是$U$的子流形,满足与$X_p$正交。$\dim S=n-1$.设$(x_1,\dots,x_{n-1})$给出$S$在$p$的坐标,$(x_1,\dots,x_{n-1},t)$给出$U$处的坐标,$\pa{t}=X_p$.

同时设$X_i=\pa{x_i}$.我们得到:
\begin{align}
  \langle \nabla_{X_j}X,X_j\rangle+\langle \nabla_{X_i}X,X_j\rangle=X\langle X_i,X_j\rangle-\langle[X,X_i],X_j\rangle-\langle [X,X_j],X_i\rangle=\pa{t}\langle X_i,X_j\rangle
\end{align}
因为$X$是Killing场,而$X_i$沿着$X$的积分曲线移动时仍为$X_i$,从而:
\begin{align}
  g(X_i(0),X_j(0))=g(\varphi_\epsilon X_i,\varphi_\epsilon X_j)=g(X_i(p),X_j(p))
\end{align}
这意味着上述式子为$0$.

反过来,若上述公式为$0$,则$g(X_i(0),X_j(0))=g(\varphi_\epsilon X_i,\varphi_\epsilon X_j)$.这意味着$g$在$X$的积分曲线上做拉回不变,因此$X$是Killing场。

(e)在(d)题下已经显然。

6.用$X$生成局部单参数变换群$\varphi_t$.则$\varphi_t(q)=0$.从而$\varphi_{t,*,q}$是$T_pM$上的线性映射。我们断言这个映射是$\mathrm{Id}$.

实际上考虑:
\begin{align}
  0=(\nabla_Y X-\nabla_X Y)(q)=[Y,X](q)=\lim_{t\to 0}\frac{1}{t}[d\varphi_{t,*,q}-\mathrm{Id}](Y)=\frac{d}{dt}(\varphi_{t,*,q})|_{t=0}
\end{align}
并且$\varphi_{t,*,q}\circ \varphi_{s,*,q}=\varphi_{t+s,*,q}$。两边对$s$求导可得:
\begin{align}
  \varphi_{s,*,q} \circ \frac{d}{dt}(\varphi_{t,*,q})|_{t=0}=\frac{d}{dt}(\varphi_{t,*,q})|_{t=s}=0
\end{align}
于是$\varphi_{t,*,q}=\mathrm{Id}$.

因为$\varphi_t$是等距,我们已经给出了其一个点处的值和微分,从而$\varphi_t$被唯一确定为$\mathrm{Id}$.

\begin{lemma}
  设$M$是完备黎曼流形,$f$是$M$的等距同构。则$f$被一个点$p$处的值和该处的微分唯一确定。
\end{lemma}
\begin{proof}
  设$q\in M$.我们说明$f(q)$只有一种选择。

  取连接$p,q$的测地线$\gamma$.于是存在$v \in T_pM$使得$q=\exp_p v$.我们断言:
  \begin{align}
    f(q)=f(\exp_p v)=\exp_{f(p)}(f_* v)
  \end{align}
  为了证明上述断言,我们说明曲线$\gamma(t)$:
  \begin{align}
    t \mapsto f(\exp_p(tv))
  \end{align}
  是从$f(p)$出发,以$f_*v$为测地线。

  求$\gamma(t)$的切向量$\dot{\gamma}(t)$:
  \begin{align}
    \dot{\gamma}(t)=f_*(\dot{\exp_p(tv)})
  \end{align}
  于是:
  \begin{align}
    \nabla_{\dot{\gamma}(t)}\dot{\gamma(t)}=f_*(\nabla_{\dot{\exp_p(tv)}}\dot{\exp_p(tv)})=0
  \end{align}
  
  上述第一个等式用到了$f_*$是等距。

  所以$f(\exp_p(tv))=\exp_{f(p)}(tf_*(v))$.带入$t=1$可得结果。
\end{proof}


7.设$M$是$n$维黎曼流形,$p\in M$.证明存在$p$在$M$中的邻域$U$和$n$个向量场$E_1,\dots,E_n\in \Gamma(X)$在$U$上的每个点都正交,使得在$p$处:
\begin{align}
  \nabla_{E_i}E_j(p)=0
\end{align}
这样的一族$E_i$被称为$p$处的局部测地标架。

\textbf{解答}:取$e_1,\dots,e_n$作为$p$处的一组单位正交基。取$U$是$p$处的一个正规坐标系,定义$E_i(\exp_p w)$是$e_i$沿着$\exp_p(tw)$平行移动得到的向量。

容易验证$E_i$正交。另外,$\nabla_{E_i}E_j(p)=\nabla_{e_i}E_j(p)=0$。因为$e_j$沿着$\exp_p(te_i)$平行移动。


8.定义向量场$X$的散度是函数$\mathrm{div}X:M \to \R$,在$p$处的值是线性映射$Y(p) \mapsto \nabla_{Y}X(p)$的迹。函数$f$的梯度则是$df$在度量下的对偶向量场。

设$E_i$是测地标架。
  
(a)证明$\mathrm{div}X(p)=\sum_{i=1}^n E_i(f_i)(p)$,其中$X=\sum_i f_i E_i$.

(b)验证上式和$M=\R^n$时在数学分析中定义的表达式相同。

\textbf{解答}:

(a)
\begin{align}
  \nabla_{E_i}(f^j E_j)=\nabla_{E_i}(f^j)E_j=E_i(f_j)E_j
\end{align}
于是该映射的迹是$\sum_i E_i(f_i)(p)$。

(b)平凡。

9.定义Laplacian算子$\Delta$为:
\begin{align}
  \Delta(f)=\mathrm{div}\mathrm{grad}f,f \in C^\infty(M)
\end{align}

\textbf{解答}:

(a)
\begin{align}
  \Delta(f)=\sum_i E_i(E_i)(p)
\end{align}

(b)
\begin{align}
  \Delta(fg)=\sum_i E_iE_i(fg)=\sum_i E_i(fE_ig+E_ifg)=2\sum_i E_ifE_ig+\sum_i E_iE_ifg+\sum_i fE_iE_ig
\end{align}

10.计算:
\begin{align}
  \frac{d}{ds}\langle \frac{\partial f}{\partial s},\frac{\partial f}{\partial t}\rangle&=\langle \frac{D}{ds}\frac{\partial f}{\partial s},\frac{\partial f}{\partial t}\rangle+\langle \frac{\partial f}{\partial s},\frac{D}{\partial t}\frac{\partial f}{\partial s}\rangle\\&=\frac{1}{2}\frac{d}{dt}\langle\frac{\partial f}{\partial s},\frac{\partial f}{\partial s}\rangle=0
\end{align}

11.对于可定向黎曼流形$\nu$,证明:
\begin{align}
  d(i_X \nu)=\mathrm{div}X \nu
\end{align}
其中$\nu$是$M$的体积形式。

\textbf{解答}:取$p \in M$和$E_i$作为测地标架。设$X=X^iE_i$.设$\omega^i \in \Omega^1(U)$满足$\omega^i(E_j)=\delta_{ij}$.

则$\omega^1\wedge \dots \omega^n$是$M$上的体积形式。于是$i_X \nu=\sum_i(-1)^{i-1}X^i \theta_i$。其中$\theta_i$是缺少$\omega^i$的$n-1$形式。
\begin{align}
  d(i_X\nu)&=\sum_i (-1)^{i-1}dX^i \wedge \theta_i+\sum_i (-1)^{i-1}X^i \wedge d\theta_i\\&=\sum_i E_i(X^i)\nu+\sum_i (-1)^{i-1}X^i \wedge d\theta_i
\end{align}

但是在$p$处$d\theta_i=0$.这是因为:
\begin{align}
  d\omega^k(E_i,E_j)=\omega^k(\nabla_{E_i} E_j-\nabla_{E_j}E_i)=0
\end{align}
再根据$p$的任意性,可知上述方程成立哦。

12.(E.Hopf定理)
\section{第四章}
\end{document}



